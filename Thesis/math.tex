\chapter{Mathematics and Science}	
\section{Math}
	\TeX\ is the best way to typeset mathematics. Donald Knuth designed \TeX\ when he got frustrated at how long it was taking the typesetters to finish his book, which contained a lot of mathematics. 
	
	If you are doing a thesis that will involve lots of math, you will want to read the following section which has been commented out. If you're not going to use math, skip over this next big red section. (It's red in the .tex file but does not show up in the .pdf.)
	
% MATH and PHYSICS majors: Uncomment the following section	
	$$\sum_{j=1}^n (\delta\theta_j)^2 \leq {\frac{\beta_i^2}{\delta_i^2 + \rho_i^2}}
\left[ 2\rho_i^2 + \frac{\delta_i^2\beta_i^2}{\delta_i^2 + \rho_i^2} \right] \equiv \omega_i^2
$$

From Informational Dynamics, we have the following (Dave Braden):

After {\it n} such encounters the posterior density for $\theta$ is

$$
\pi(\theta|X_1< y_1,\dots,X_n<y_n) \varpropto \pi(\theta) \prod_{i=1}^n\int_{-\infty}^{y_i}
   \exp\left(-\frac{(x-\theta)^2}{2\sigma^2}\right)\ dx
$$



Another equation:

$$\det\left|\,\begin{matrix}%
c_0&c_1\hfill&c_2\hfill&\ldots&c_n\hfill\cr
c_1&c_2\hfill&c_3\hfill&\ldots&c_{n+1}\hfill\cr
c_2&c_3\hfill&c_4\hfill&\ldots&c_{n+2}\hfill\cr
\,\vdots\hfill&\,\vdots\hfill&
  \,\vdots\hfill&&\,\vdots\hfill\cr
c_n&c_{n+1}\hfill&c_{n+2}\hfill&\ldots&c_{2n}\hfill\cr
\end{matrix}\right|>0$$


Lapidus and Pindar, Numerical Solution of Partial Differential Equations in Science and
Engineering.  Page 54

$$
\int_t\left\{\sum_{j=1}^3 T_j \left(\frac{d\phi_j}{dt}+k\phi_j\right)-kT_e\right\}w_i(t)\ dt=0,
   \qquad\quad i=1,2,3. 
$$

L\&P  Galerkin method weighting functions.  Page 55

$$
\sum_{j=1}^3 T_j\int_0^1\left\{\frac{d\phi_j}{dt} + k\phi_j\right\} \phi_i\ dt 
   = \int_{0}^1k\,T_e\phi_idt, \qquad i=1,2,3 $$
   
Another L\&P (p145)

$$
\int_{-1}^1\!\int_{-1}^1\!\int_{-1}^1 f\big(\xi,\eta,\zeta\big) 
   = \sum_{k=1}^n\sum_{j=1}^n\sum_{i=1}^n w_i w_j w_k f\big( \xi,\eta,\zeta\big).
$$

Another L\&P (p126)

$$
\int_{A_e} (\,\cdot\,) dx dy = \int_{-1}^1\!\int_{-1}^1 (\,\cdot\,) \det[J] d\xi d\eta.
$$

\section{Chemistry 101: Symbols}
Chemical formulas will look best if they are not italicized. Get around math mode's automatic italicizing by using the argument \verb=$\mathrm{formula here}$=, with your formula inside the curly brackets.

So, $\mathrm{Fe_2^{2+}Cr_2O_4}$ is written \verb=$\mathrm{Fe_2^{2+}Cr_2O_4}$=\\
Exponent or Superscript: O$^{-}$\\
Subscript: CH$_{4}$\\

To stack numbers or letters as in $\mathrm{Fe_2^{2+}}$, the subscript is defined first, and then the superscript is defined.\\
Angstrom: {\AA}\\
Bullet: CuCl $\bullet$ 7H${_2}$O\\
Double Dagger: \ddag \/\\
Delta: $\Delta$\\
Reaction Arrows: $\longrightarrow$ or  $\xrightarrow{solution}$\\
Resonance Arrows: $\leftrightarrow$\\
Reversible Reaction Arrows: $\rightleftharpoons$ or $\xrightleftharpoons[ ]{solution}$ (the latter requires the chemarr package)\\


\subsection{Typesetting reactions}
You may wish to put your reaction in a figure environment, which means that LaTeX will place the reaction where it fits and you can have a figure legend if desired:
\begin{figure}[htbp]
\begin{center}
$\mathrm{C_6H_{12}O_6  + 6O_2} \longrightarrow \mathrm{6CO_2 + 6H_2O}$
\caption{Combustion of glucose}
\label{combustion of glucose}
\end{center}
\end{figure}

\subsection{Other examples of reactions}
$\mathrm{NH_4Cl_{(s)}} \rightleftharpoons \mathrm{NH_{3(g)}+HCl_{(g)}}$\\
$\mathrm{MeCH_2Br + Mg} \xrightarrow[below]{above} \mathrm{MeCH_2\bullet Mg \bullet Br}$

\section{Physics}

Many of the symbols you will need can be found on the math page (\url{http://web.reed.edu/cis/help/latex/math.html}) and the Comprehensive \LaTeX\ Symbol Guide (enclosed in this template download).  You may wish to create custom commands for commonly used symbols, phrases or equations, as described in Chapter \ref{commands}.

\section{Biology}
You will probably find the resources at \url{http://www.lecb.ncifcrf.gov/~toms/latex.html} helpful, particularly the links to bsts for various journals. You may also be interested in TeXShade for nucleotide typesetting (\url{http://homepages.uni-tuebingen.de/beitz/txe.html}).  Be sure to read the proceeding chapter on graphics and tables, and remember that the thesis template has versions of Ecology and Science bsts which support webpage citation formats. 