\chapter{Numerics and Workflow}

    	 	\epigraph{On two occasions I have been asked, "Pray, Mr. Babbage, if you put into the machine wrong figures, will the right answers come out?" ... I am not able rightly to apprehend the kind of confusion of ideas that could provoke such a question.}{Charles Babbage, Passages from the Life of a Philosopher}

Computational studies of Navier-Stokes, especially as a dynamical system is difficult for many reasons, but most important among them is the high degree of complexity inherent in the numerical tools required to efficiently simulate it. As mentioned earlier, the two major approaches towards simulating Navier-Stokes  is modelling, where some assumptions are made to reduce the difficult of simulation, and direct numerical simulation (DNS), where no assumptions beyond those used to derive Navier-Stokes and set up the boundary conditions are used. DNS is naturally more accurate (since the physicality of some modelling assumptions can be suspect), but since it fully resolves Navier-Stokes, it is significantly more expensive than modeling, and methods that attempt to offset this extra cost tend to be extremely complex. For this reason, I use the open source library {\tt Channelflow}\rf{Gibson2014}, which has been essential in making any headway in this thesis. {\tt Channelflow} is a spectral DNS library, with additional utilities to find, parametrically continue, and analyze \ecs, which I will lay out in some detail below.
\section{The Spectral Method}
\subsection{The Residual}

Spectral methods are, like finite element methods, part of a larger class of numerical methods known as {\bf weighted residual methods}. In this class of methods, functions are approximated by a truncated series expansion, with the restriction that a quantity related to the residual be zero (instead of the residual itself). The quantity used for the sprectral method is the scalar product 
\begin{equation}\label{eq:scalarProduct}
\scprod{u}{v}{w} = \int{a}{b}{uvw}{dx},
\end{equation}
where  $u(x),v(x)$ are some functions on the interval $[a,b]$, and $w(x)$ is a weighting function. If we then imagine some platonic function\footnote{That is, the function that is to be approximated} $u(x)$ that we attempt to approximate via a finite series expansion, so that
\begin{equation}\label{eq:seriesExpansion}
u_N(x) = \sum{k=0}{N}{\hat{u}_k \psi_k(x)},
\end{equation}
for some set of orthogonal basis functions $\psi_k(x)$, then the residual is given by
\begin{equation}
R_N = u(x)-u_N(x),
\end{equation} 
and for some differential equation
\begin{equation}
Du = f,
\end{equation}
where $D$ is some arbitrary differential operator and $f(x)$ is some arbitrary source function, the residual is defined
\begin{equation}
R_N = Du_N,
\end{equation}
as expected. While it may seem logical to require $R_N = 0$, this is not possible in general for finite $N$, so we instead require 
\begin{equation}
\scprod{R_N}{\delta(x-x_i)}{1} = R_N(x_i) = 0,
\end{equation}
for some set of $x_i$. This method of minimizing the residual is known as the collocation method, has the effect of forcing the series expansion to match the platonic function at some set of points.\footnote{The other popular method of minimizing the residual, the Galerkin method, sets the mean residual as zero} In the spectral method, the basis functions are trignometric functions. The advantage of trignometric functions over polynomials is the rapid convergence of the series coefficients -- the Fourier coefficients converge exponentially fast, so we can achieve extremely high accuracy with a lower number of modes\rf{Peyret2002}.\footnote{Spectral methods are inappropriate when the boundary geometry is highly complex, as is the case in the majority of industrial applications, where more general finite element methods are used.} {\tt Channelflow} uses Fourier series, where the basis functions are given by 
\begin{equation}
F_k(x) = \exp{ikx},
\end{equation}
and Chebyshev polynomials of the first kind, where the basis functions satisfy
\begin{equation}
T_k(x) = \cos{k\arccos{x}}.
\end{equation}
The Fourier series expansion is particularly nice since the derivative $\partial_x F_k(x) = ikF(x)$. To see how the spectral method is applied to solving nonlinear partial differential equations, consider the 1D Swift-Hohenburg equation
\begin{equation}\label{eq:SHOG}
\pder{1}{u}{t} = \kappa u - \paren{1+\pder{2}{}{x}}^2u-u^2.
\end{equation}
Expanding \refeq{eq:SHOG}, we get 
\begin{equation}\label{eq:SHEX}
\pder{1}{u}{t} = \paren{\kappa -1}u - 2\pder{2}{u}{x}-\pder{4}{u}{x} - u^2.
\end{equation}
If we assume a Fourier series approximation to $u$, and denote the approximation at time $n\Delta t$ by $u_K^n(x)$, then using a forward-difference method to discretize \refeq{eq:SHEX}, we get the residual 
\begin{equation}
R_K = \frac{u_K^{n+1}-u_K^n}{\Delta t} + (\kappa-1)u_K^n 
\end{equation}
While Fourier expansions are the easiest to deal with, they are best used on periodic boundaries, since the imposition of aperiodic boundary conditions on a Fourier series expansion can lead to Gibbs oscillations that can make the simulation aphysical\rf{Peyret2002}. For this reason, the velocity field is expanded as a Fourier series in the plane, and as a Chebyshev polynomial in the wall-normal direction. The velocity field is then represented as 