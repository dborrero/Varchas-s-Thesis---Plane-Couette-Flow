
%%%% REFERENCES %%%%%
\newcommand{\rf}     [1] {~\cite{#1}}
\newcommand{\refref} [1] {ref.~\cite{#1}}
\newcommand{\refRef} [1] {Ref.~\cite{#1}}
\newcommand{\refrefs}[1] {refs.~\cite{#1}}
\newcommand{\refRefs}[1] {Refs.~\cite{#1}}
\newcommand{\refeq}  [1] {(\ref{#1})}
\newcommand{\refeqs} [2]{(\ref{#1}--\ref{#2})}
\newcommand{\reffig} [1] {figure~\ref{#1}}
\newcommand{\reffigs} [2] {figures~\ref{#1} and~\ref{#2}}
\newcommand{\refFig} [1] {Figure~\ref{#1}}
\newcommand{\refFigs} [2] {Figures~\ref{#1} and~\ref{#2}}
\newcommand{\reftab} [1] {table~\ref{#1}}
\newcommand{\refTab} [1] {Table~\ref{#1}}
\newcommand{\reftabs}[2] {tables~\ref{#1} and~\ref{#2}}
\newcommand{\refsect}[1] {Section~\ref{#1}}
\newcommand{\refsects}[2] {Sections~\ref{#1} and \ref{#2}}
\newcommand{\refSect}[1] {Sect.~\ref{#1}}
\newcommand{\refSects}[2] {Sects.~\ref{#1} and \ref{#2}}
\newcommand{\refsecttosect}[2] {Sects.~\ref{#1} to~\ref{#2}}
\newcommand{\refappe}[1] {appendix~\ref{#1}}
\newcommand{\refappes}[2] {appendices~\ref{#1} and~\ref{#2}}
\newcommand{\refAppe}[1] {Appendix~\ref{#1}}
\newcommand{\refChapter}[1]{Chapter~\ref{#1}}
\newcommand{\refChapt}[1]{Chapt.~\ref{#1}}
\newcommand{\refDef}[1]{Definition~\ref{#1}}
%%%% SYMBOLS %%%%
\newcommand{\ReN}{\ensuremath{Re}} % Reynolds number
\newcommand{\pCf}{plane Couette flow}
\newcommand{\PCf}{Plane Couette flow} % plane Couette flow
\newcommand{\ecs}{exact coherent structures}
\newcommand{\Ecs}{Exact coherent structures}
%%%% ABBREVIATIONS %%%%%
\newcommand{\etc}{{etc.}}       % APS
\newcommand{\etal}{{\em et al.}}    % etal in italics, APS too
\newcommand{\ie}{{i.e.}}        % APS
\newcommand{\cf}{{\em cf.\ }}     % APS
\newcommand{\eg}{{e.g.\ }}        % APS, OUP, hard space '\eg\ NextWord'

%%%%% EDITING COMMANDS %%%%%
\newcommand{\DB}[2]{$\footnotemark\footnotetext{DB #1: {\color{red}#2}}$} %date, comment
\newcommand{\DBedit}[1]{{\color{red}#1}}
\newcommand{\VGC}[2]{$\footnotemark\footnotetext{DB #1: {\color{blue}#2}}$} %date, comment
\newcommand{\VGedit}[1]{{\color{blue}#1}}


%%%% SUPER USEFUL COMMANDS THAT I'M USED TO HAVING%%%%%
\newtheorem{define}{Definition}
\let\Oldfrac\frac
\renewcommand{\frac}[2]{\dfrac{#1}{#2}}

\let\Oldsin\sin
\renewcommand{\sin}[1]{\Oldsin{\left ( #1  \right ) }}

\let\Oldcos\cos
\renewcommand{\cos}[1]{\Oldcos{\left ( #1  \right ) }}

\newcommand{\abs}[1]{\left | #1 \right |}
\newcommand{\sqfrac}[2]{\sqrt{\frac{#1}{#2}}}
\newcommand{\paren}[1]{\left ( #1 \right )}
\newcommand{\sbrac}[1]{\left [ #1 \right ]}
\newcommand{\scprod}[3]{\left < #1,#2 \right >_{#3}}

\let\Oldint\int
\renewcommand{\int}[4]{\Oldint_{#1}^{#2} #3 \hspace{2mm}#4}
\let\Oldsum\sum
\renewcommand{\sum}[3]{\Oldsum\limits_{#1}^{#2} #3}


\newcommand{\pder}[3]{\ifnum#1=1
							\dfrac{\partial#2}{\partial#3}
					   \else
					   \ifnum#1>1\dfrac{\partial^{#1}#2}{\partial#3^{#1}} \fi \fi }
\newcommand{\der}[3]{\ifnum#1=1
							\dfrac{\text{d}#2}{\text{d}#3}
					   \else
					   \ifnum#1>1\dfrac{\text{d}^{#1}#2}{\text{d}#3^{#1}} \fi \fi }
\newcommand{\Vector}[1]{\mathbf{#1}}
\newcommand{\Tensor}[1]{\mathcal{#1}}
\newcommand{\function}[2]{#1\!\paren{#2}}

\newcommand{\Div}[1]{\nabla\cdot#1}
\newcommand{\Grad}[1]{\nabla\,#1}
\newcommand{\equationref}[1]{Equation~\ref{#1}}
\newcommand{\figureref}[1]{Figure~\ref{#1}}