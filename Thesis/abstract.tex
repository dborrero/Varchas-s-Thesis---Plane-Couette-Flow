% The abstract is not required if you're writing a creative thesis (but aren't they all?)
% If your abstract is longer than a page, there may be a formatting issue.
    \chapter*{Abstract}
	\Ecs\ are an exciting and potentially revolutionary method of understanding the twin problems of turbulent dynamics and the transition to turbulence. Exact coherent structures are invariant solutions of the fully resolved Navier-Stokes equation.  In \pCf, the flow between infinite shearing plates, the inherent symmetries of the problem lead to symmetric \ecs, which are computationally easier to find. However, turbulence itself is a fundamentally asymmetric phenomenon, and may be better described by \ecs\ with broken symmetry. In this thesis, we report the discovery of four new periodic orbits -- P85 and P60 which are fully symmetric, and P32 and P8, which have partially broken symmetry. Direct numerical simulation was done using the computational fluid dynamics library {\tt Channelflow}. Comparison of the projections of these periodic orbits in the dissipation-energy input plane with randomly seeded turbulent trajectories reveals that P32, P60 and P85 lie in the turbulent region of the state space, while P8 lies very far away from this region. Nevertheless, we focus on P8 so as to best utilize our limited computational resources. Parametric continuation in the spanwise periodic cell length $L_z$ suggests that P8 undergoes two bifurcations. This is verified by analysis of various properties of P8 in the dissipation-energy input plane and by observations of a changes in the stability of eigenvectors that are consistent with bifurcations.   