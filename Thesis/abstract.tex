% The abstract is not required if you're writing a creative thesis (but aren't they all?)
% If your abstract is longer than a page, there may be a formatting issue.
    \chapter*{Abstract}
	\Ecs are an exciting and potentially revolutionary method of understanding turbulent dynamics and transition. In \pCf, the inherent symmetries naturally lead to symmetric \ecs, which are computationally easier to find. However, turbulence itself is a fundamentally asymmetric phenomenon, and may be better described by symmetry broken \ecs. In this thesis, we find four new periodic orbits -- P85 and P60m which have unbroken symmetry and P32 and P8, which have partially broken symmetry, using the computational fluid dynamics library {\tt Channelflow}. Comparison of the projections of these periodic orbits in the dissipation-energy input plane with randomly seeded turbulent trajectories reveals that P32, P60 and P85 lie in the turbulent region, while P8 lies very far away from the turbulent region. Nevertheless, we focus on P8 so as to best utilize limited computational resources. Parametric continuation in the spanwise periodic cell length $L_z$ suggests that P8 undergoes two bifurcations. This is verified by analysis of various properties of P8 in the dissipation-energy input plane, as well as an observation of a change of stability of eigenvectors that are consistent with the bifurcation.   