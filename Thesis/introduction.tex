%The \introduction command is provided as a convenience.
%if you want special chapter formatting, you'll probably want to avoid using it altogether
		
\chapter*{Introduction}
    \addcontentsline{toc}{chapter}{Introduction}
		\chaptermark{Introduction}
		\markboth{Introduction}{Introduction}
% The three lines above are to make sure that the headers are right, that the intro gets included in the table of contents, and that it doesn't get numbered 1 so that chapter one is 1.
\epigraph{ I am an old man now, and when I die and go to heaven, there are two matters on which I hope for enlightenment. One is quantum electrodynamics, and the other is the turbulent motion of fluids. And about the former, I am optimistic}{Horace Lamb, 1932}
	Among the many successes of classical mechanics since Newton's \emph{Principia}, the modern reader may notice one conspicuous absence -  turbulence. Like a certain small, unnamed costal village in Armorica holding out against the Romans, turbulence has confounded the best efforts of physicists and engineers for over a century, with no real end in sight. \\
	
	Historical approaches towards understanding turbulence began via a statistical approach, describing it as a random perturbation about some mean flow, resulting in Kolmogorov's famous scaling laws in 1941~\cite{Kolmogorov1991}, and von Karman's so-called 'law of the wall' in 1930~\cite{Karman1930}. These approaches, while perfectly suited to model the average behavior of turbulent flow, nevertheless fails to capture the dynamic behavior that would be the holy grail of fluid dynamics. The hope of the dynamic systems theory approach, lead by Hopf, Poincare and many other physicists is that this behavior can, to some extent, be captured in a meaningful way. 
	
	\section{Dynamical Systems and Hopf's Dream} 
	
The prequel to Hopf and turbulence begins, somewhat unsurprisingly, with Newton. Newton showed in the \emph{Principia} that his law of gravitation was consistent with his observations by solving the two-body problem, a procedure that is now routine in undergraduate classical mechanics courses. However, he was unable to solve the three-body problem - and neither was Gauss, Euler, d'Alembert or any other mathematical titan of the 18th and 19th centuries. In 1885, perhaps slightly frustrated with the unwillingness of nature to play ball with humans, King Oskar of Sweden announced a prize to the first person to the first person to find an exact analytic solution to the problem. Unfortunately for him, and for anyone else hoping for a tidy solution to the problem, Henri Poincare showed in 1887 that no general analytic solution existed for the problem (or, by extension for the n-body problem where $n>2$), but fortunately for students of dynamical system's theory, setting the foundations for the geometric approach that is still used today. This is where Hopf comes in. Hopf had a vision of fluid flow as a vector in a infinite dimensional phase space, with viscosity forcing the phase space trajectory of the flow to lie in some finite manifold in the long term~\cite{Hopf1948}. Hopf provides, as an example. the case where the viscosity $\mu$ is very large, in which case the manifold shrinks to a point corresponding to laminar flow, and speculates that as the viscosity decreases, new manifolds should arise from bifurcations, envisioning a maze of recurrent manifolds springing forth from the aether. Sadly for Hopf, the first electronic computer, ENIAC, had been built just two years earlier, the first silicon transistor was six years in the future, and high performance computing still decades away. With the resources available to him at the time, a numerical simulation of this phase space was impossible (to say nothing of analytic solutions), and Hopf had to remain satisfied with applying his ideas to approximations of the Navier-Stokes equations. \\

	\section{Computers and the Future}
	
With the advent of modern computing however, numerical simulation of the phase space topology is within reach. In plane Couette flow (described in more detail in Section), cartographic efforts began with Nagata's demonstration of the existence of finite-amplitude turbulent perturbations from mean flow that were nevertheless equilibrium solutions in 1990~\cite{Nagata1990} and Kawahara and Kida's determination of \emph{periodic} turbulent perturbations from mean flow~\cite{Kawahara2001} in 2001, and Viswanath's calculation of \emph{relative} periodic orbits in 2007~\cite{Viswanath2007}, which also introduced the numerical scheme that constitutes one of the core solvers used in this thesis. The development of the Channelflow software library by Gibson~\cite{Gibson2008}~\cite{Gibson2014} is of particular note, as it has enabled the wider investigation of the phase space topology, and features heavily in this thesis. Indeed,  the numerical schemes used within are formidable, and certainly beyond my ability to recreate within the thesis timescale, though I shall outline them in Section~\ref{sec:channelflow}. Given these tools, then, we can imagine that it may be possible to construct a web of periodic orbits, equilibria and their heteroclinc connections, and then predict to some accuracy the long-term dynamical behavior, based on transitions between these different states. 