% This is the Reed College LaTeX thesis template. Most of the work 
% for the document class was done by Sam Noble (SN), as well as this
% template. Later comments etc. by Ben Salzberg (BTS). Additional
% restructuring and APA support by Jess Youngberg (JY).
% Your comments and suggestions are more than welcome; please email
% them to cus@reed.edu
%
% See http://web.reed.edu/cis/help/latex.html for help. There are a 
% great bunch of help pages there, with notes on
% getting started, bibtex, etc. Go there and read it if you're not
% already familiar with LaTeX.
%
% Any line that starts with a percent symbol is a comment. 
% They won't show up in the document, and are useful for notes 
% to yourself and explaining commands. 
% Commenting also removes a line from the document; 
% very handy for troubleshooting problems. -BTS

% As far as I know, this follows the requirements laid out in 
% the 2002-2003 Senior Handbook. Ask a librarian to check the 
% document before binding. -SN

%%
%% Preamble
%%
% \documentclass{<something>} must begin each LaTeX document
\documentclass[12pt,twoside]{reedthesis}
% Packages are extensions to the basic LaTeX functions. Whatever you
% want to typeset, there is probably a package out there for it.
% Chemistry (chemtex), screenplays, you name it.
% Check out CTAN to see: http://www.ctan.org/
%%
\usepackage{graphicx,latexsym} 
\usepackage{amssymb,amsthm,amsmath}
\usepackage{longtable,booktabs,setspace} 
\usepackage{chemarr} %% Useful for one reaction arrow, useless if you're not a chem major
\usepackage[hyphens]{url}
\usepackage{rotating}
\usepackage{natbib}
\usepackage{ifthen}
\graphicspath{{./Figs/}} % Set graphics path
% Comment out the natbib line above and uncomment the following two lines to use the new 
% biblatex-chicago style, for Chicago A. Also make some changes at the end where the 
% bibliography is included. 
%\usepackage{biblatex-chicago}
%\bibliography{thesis}

% \usepackage{times} % other fonts are available like times, bookman, charter, palatino


%%%% REFERENCES %%%%%
\newcommand{\rf}     [1] {~\cite{#1}}
\newcommand{\refref} [1] {ref.~\cite{#1}}
\newcommand{\refRef} [1] {Ref.~\cite{#1}}
\newcommand{\refrefs}[1] {refs.~\cite{#1}}
\newcommand{\refRefs}[1] {Refs.~\cite{#1}}
\newcommand{\refeq}  [1] {(\ref{#1})}
\newcommand{\refeqs} [2]{(\ref{#1}--\ref{#2})}
\newcommand{\reffig} [1] {figure~\ref{#1}}
\newcommand{\reffigs} [2] {figures~\ref{#1} and~\ref{#2}}
\newcommand{\refFig} [1] {Figure~\ref{#1}}
\newcommand{\refFigs} [2] {Figures~\ref{#1} and~\ref{#2}}
\newcommand{\reftab} [1] {table~\ref{#1}}
\newcommand{\refTab} [1] {Table~\ref{#1}}
\newcommand{\reftabs}[2] {tables~\ref{#1} and~\ref{#2}}
\newcommand{\refsect}[1] {Section~\ref{#1}}
\newcommand{\refsects}[2] {Sections~\ref{#1} and \ref{#2}}
\newcommand{\refSect}[1] {Sect.~\ref{#1}}
\newcommand{\refSects}[2] {Sects.~\ref{#1} and \ref{#2}}
\newcommand{\refsecttosect}[2] {Sects.~\ref{#1} to~\ref{#2}}
\newcommand{\refappe}[1] {appendix~\ref{#1}}
\newcommand{\refappes}[2] {appendices~\ref{#1} and~\ref{#2}}
\newcommand{\refAppe}[1] {Appendix~\ref{#1}}
\newcommand{\refChapter}[1]{Chapter~\ref{#1}}
\newcommand{\refChapt}[1]{Chapt.~\ref{#1}}

%%%% SYMBOLS %%%%
\newcommand{\ReN}{\ensuremath{Re}} % Reynolds number
\newcommand{\pCf}{plane Couette flow}
\newcommand{\PCf}{Plane Couette flow} % plane Couette flow
\newcommand{\ecs}{exact coherent structures}
\newcommand{\Ecs}{Exact coherent structures}
%%%% ABBREVIATIONS %%%%%
\newcommand{\etc}{{etc.}}       % APS
\newcommand{\etal}{{\em et al.}}    % etal in italics, APS too
\newcommand{\ie}{{i.e.}}        % APS
\newcommand{\cf}{{\em cf.\ }}     % APS
\newcommand{\eg}{{e.g.\ }}        % APS, OUP, hard space '\eg\ NextWord'

%%%%% EDITING COMMANDS %%%%%
\newcommand{\DB}[2]{$\footnotemark\footnotetext{DB #1: {\color{red}#2}}$} %date, comment
\newcommand{\DBedit}[1]{{\color{red}#1}}
\newcommand{\VGC}[2]{$\footnotemark\footnotetext{DB #1: {\color{blue}#2}}$} %date, comment
\newcommand{\VGedit}[1]{{\color{blue}#1}}


%%%% SUPER USEFUL COMMANDS THAT I'M USED TO HAVING%%%%%
\let\Oldfrac\frac
\renewcommand{\frac}[2]{\dfrac{#1}{#2}}

\let\Oldsin\sin
\renewcommand{\sin}[1]{\Oldsin{\left ( #1  \right ) }}

\let\Oldcos\cos
\renewcommand{\cos}[1]{\Oldcos{\left ( #1  \right ) }}

\newcommand{\abs}[1]{\left | #1 \right |}
\newcommand{\sqfrac}[2]{\sqrt{\frac{#1}{#2}}}
\newcommand{\paren}[1]{\left ( #1 \right )}
\newcommand{\sbrac}[1]{\left [ #1 \right ]}
\newcommand{\scprod}[3]{\left < #1,#2 \right >_{#3}}

\let\Oldint\int
\renewcommand{\int}[4]{\Oldint_{#1}^{#2} #3 \hspace{2mm}#4}
\let\Oldsum\sum
\renewcommand{\sum}[3]{\Oldsum\limits_{#1}^{#2} #3}


\newcommand{\pder}[3]{\ifnum#1=1
							\dfrac{\partial#2}{\partial#3}
					   \else
					   \ifnum#1>1\dfrac{\partial^{#1}#2}{\partial#3^{#1}} \fi \fi }
\newcommand{\der}[3]{\ifnum#1=1
							\dfrac{\text{d}#2}{\text{d}#3}
					   \else
					   \ifnum#1>1\dfrac{\text{d}^{#1}#2}{\text{d}#3^{#1}} \fi \fi }
\newcommand{\Vector}[1]{\mathbf{#1}}
\newcommand{\Tensor}[1]{\mathcal{#1}}
\newcommand{\function}[2]{#1\!\paren{#2}}

\newcommand{\Div}[1]{\nabla\cdot#1}
\newcommand{\Grad}[1]{\nabla\,#1}
\newcommand{\equationref}[1]{Equation~\ref{#1}}
\newcommand{\figureref}[1]{Figure~\ref{#1}} % Load thesis specific macros

\title{My Final College Paper}
\author{Your R. Name}
% The month and year that you submit your FINAL draft TO THE LIBRARY (May or December)
\date{May 200x}
\division{Mathematics and Natural Sciences}
\advisor{Advisor F. Name}
%If you have two advisors for some reason, you can use the following
%\altadvisor{Your Other Advisor}
%%% Remember to use the correct department!
\department{Mathematics}
% if you're writing a thesis in an interdisciplinary major,
% uncomment the line below and change the text as appropriate.
% check the Senior Handbook if unsure.
%\thedivisionof{The Established Interdisciplinary Committee for}
% if you want the approval page to say "Approved for the Committee",
% uncomment the next line
%\approvedforthe{Committee}

\setlength{\parskip}{0pt}
%%
%% End Preamble
%%
%% The fun begins:
\begin{document}

  \maketitle
  \frontmatter % this stuff will be roman-numbered
  \pagestyle{empty} % this removes page numbers from the frontmatter

% Acknowledgements (Acceptable American spelling) are optional
% So are Acknowledgments (proper English spelling)
	% Acknowledgements (Acceptable American spelling) are optional
% So are Acknowledgments (proper English spelling)
    \chapter*{Acknowledgments}
	\epigraph{Some people are of the opinion that acknowledgments ought to be concise, relevant to the thesis, and devoid of sappy sentiment. They are probably right, but they aren't the ones writing this.}{Varchas Gopalaswamy}

This thesis wouldn't have existed without my thesis advisor, Daniel Borrero. Thank you, Daniel, for suggesting an awesome thesis topic, for motivating me, for spending a ton of time correcting more terrible thesis drafts than any human should have to, for introducing me to dynamical systems theory, and for all the grad school help. Good luck with setting up the Taylor-Couette system.\\

I would also like to thank John Gibson from the University of New Hampshire for making this thesis even remotely feasible by writing {\tt Channelflow}, and for taking the time to correspond with me via email and Hangout. Your advice has been invaluable. \\
 
By definition, this thesis wouldn't have existed were I not a physics major, so I'd like to thank the department as a whole for being the greatest department at Reed. Thank you, Lucas, for being an extremely supportive academic advisor and for a  challenging junior year. Thank you, Joel, for introducing me to the world of scientific computation, for running physics softball, and for your help with this thesis. Thank you, Darrell, for reminding me why quantum mechanics is awesome. Thank you, Johnny, for all the help with the grad school process. Thanks also to the physics seniors -- shoutout to Julia and Neal for their tenure as the Pub Czars, Taras for his tenure as the Cookie Czar, Dan for putting up with the frantic late night calls for help, and Newton for being a rad office buddy.  I will be proud to say that I was once a part of this group. You guys have made my time here special. \\

To Amma and Appa -- Thank you for all your love and support throughout all these years, and always being understanding and there for me when I have the occasional mental breakdown. Also, thanks for going ``Yes, let's send Varchas to this college we've literally never heard anything about, this sounds like a great idea." I hope you feel that you've made the right decision. 

Dear Medha: Thank you for allowing me to lecture at you about stuff you may or may not actually be interested in, for enabling me in some truly horrendous jokes, for doing all the housework whenever I would come back for vacation, and generally being an awesome sister. \\

To all my friends -- Thank you for making Reed the best of times, and for supporting me through the worst of times. Words cannot express how much I will miss you all.\footnote{Yes Matt, even you.}






% The preface is optional
% To remove it, comment it out or delete it.
	% The preface is optional
% To remove it, comment it out or delete it.
    \chapter*{Preface}
	This is an example of a thesis setup to use the reed thesis document class.

  \tableofcontents
% if you want a list of tables, optional
  \listoftables
% if you want a list of figures, also optional
  \listoffigures
		
% The abstract is not required if you're writing a creative thesis (but aren't they all?)
% If your abstract is longer than a page, there may be a formatting issue.
	% The abstract is not required if you're writing a creative thesis (but aren't they all?)
% If your abstract is longer than a page, there may be a formatting issue.
    \chapter*{Abstract}
	\Ecs\ are an exciting and potentially revolutionary method of understanding the twin problems of turbulent dynamics and the transition to turbulence. Exact coherent structures are invariant solutions of the fully resolved Navier-Stokes equation.  In \pCf, the flow between infinite shearing plates, the inherent symmetries of the problem lead to symmetric \ecs, which are computationally easier to find. However, turbulence itself is a fundamentally asymmetric phenomenon, and may be better described by \ecs\ with broken symmetry. In this thesis, we report the discovery of four new periodic orbits -- P85 and P60 which are fully symmetric, and P32 and P8, which have partially broken symmetry. Direct numerical simulation was done using the computational fluid dynamics library {\tt Channelflow}. Comparison of the projections of these periodic orbits in the dissipation-energy input plane with randomly seeded turbulent trajectories reveals that P32, P60 and P85 lie in the turbulent region of the state space, while P8 lies very far away from this region. Nevertheless, we focus on P8 so as to best utilize our limited computational resources. Parametric continuation in the spanwise periodic cell length $L_z$ suggests that P8 undergoes two bifurcations. This is verified by analysis of various properties of P8 in the dissipation-energy input plane and by observations of a changes in the stability of eigenvectors that are consistent with bifurcations.   
		\chapter*{Dedication}
\centerline{
{\Huge {\devnf %
s{\char40}ym\kRn{-0.850}{\char3}\kRn{0.050}v %
jyt\kRn{-0.850}{\char3}\kRn{0.050} }% End of Indic Script text.
}
}
	To Perfect Woodbridge \\

  \mainmatter % here the regular arabic numbering starts
  \pagestyle{fancyplain} % turns page numbering back on

% Double spacing: if you want to double space, or one and a half 
% space, uncomment one of the following lines. You can go back to 
% single spacing with the \singlespacing command.
% \onehalfspacing
% \doublespacing

	%The \introduction command is provided as a convenience.
%if you want special chapter formatting, you'll probably want to avoid using it altogether
		
\chapter*{Introduction}
    \addcontentsline{toc}{chapter}{Introduction}
		\chaptermark{Introduction}
		\markboth{Introduction}{Introduction}
% The three lines above are to make sure that the headers are right, that the intro gets included in the table of contents, and that it doesn't get numbered 1 so that chapter one is 1.
\epigraph{ I am an old man now, and when I die and go to heaven, there are two matters on which I hope for enlightenment. One is quantum electrodynamics, and the other is the turbulent motion of fluids. And about the former, I am optimistic}{Horace Lamb, 1932}
	Among the many successes of classical mechanics since Newton's \emph{Principia}, the modern reader may notice one conspicuous absence -  turbulence. Like a certain small, unnamed costal village in Armorica holding out against the Romans, turbulence has confounded the best efforts of physicists and engineers for over a century, with no real end in sight. \\
	
	Historical approaches towards understanding turbulence began via a statistical approach, describing it as a random perturbation about some mean flow, resulting in Kolmogorov's famous scaling laws in 1941~\cite{Kolmogorov1991}, and von Karman's so-called 'law of the wall' in 1930~\cite{Karman1930}. These approaches, while perfectly suited to model the average behavior of turbulent flow, nevertheless fails to capture the dynamic behavior that would be the holy grail of fluid dynamics. The hope of the dynamic systems theory approach, lead by Hopf, Poincare and many other physicists is that this behavior can, to some extent, be captured in a meaningful way. 
	
	\section{Dynamical Systems and Hopf's Dream} 
	
The prequel to Hopf and turbulence begins, somewhat unsurprisingly, with Newton. Newton showed in the \emph{Principia} that his law of gravitation was consistent with his observations by solving the two-body problem, a procedure that is now routine in undergraduate classical mechanics courses. However, he was unable to solve the three-body problem - and neither was Gauss, Euler, d'Alembert or any other mathematical titan of the 18th and 19th centuries. In 1885, perhaps slightly frustrated with the unwillingness of nature to play ball with humans, King Oskar of Sweden announced a prize to the first person to the first person to find an exact analytic solution to the problem. Unfortunately for him, and for anyone else hoping for a tidy solution to the problem, Henri Poincare showed in 1887 that no general analytic solution existed for the problem (or, by extension for the n-body problem where $n>2$), but fortunately for students of dynamical system's theory, setting the foundations for the geometric approach that is still used today. This is where Hopf comes in. Hopf had a vision of fluid flow as a vector in a infinite dimensional phase space, with viscosity forcing the phase space trajectory of the flow to lie in some finite manifold in the long term~\cite{Hopf1948}. Hopf provides, as an example. the case where the viscosity $\mu$ is very large, in which case the manifold shrinks to a point corresponding to laminar flow, and speculates that as the viscosity decreases, new manifolds should arise from bifurcations, envisioning a maze of recurrent manifolds springing forth from the aether. Sadly for Hopf, the first electronic computer, ENIAC, had been built just two years earlier, the first silicon transistor was six years in the future, and high performance computing still decades away. With the resources available to him at the time, a numerical simulation of this phase space was impossible (to say nothing of analytic solutions), and Hopf had to remain satisfied with applying his ideas to approximations of the Navier-Stokes equations. \\

	\section{Computers and the Future}
	
With the advent of modern computing however, numerical simulation of the phase space topology is within reach. In plane Couette flow (described in more detail in Section), cartographic efforts began with Nagata's demonstration of the existence of finite-amplitude turbulent perturbations from mean flow that were nevertheless equilibrium solutions in 1990~\cite{Nagata1990} and Kawahara and Kida's determination of \emph{periodic} turbulent perturbations from mean flow~\cite{Kawahara2001} in 2001, and Viswanath's calculation of \emph{relative} periodic orbits in 2007~\cite{Viswanath2007}, which also introduced the numerical scheme that constitutes one of the core solvers used in this thesis. The development of the Channelflow software library by Gibson~\cite{Gibson2008}~\cite{Gibson2014} is of particular note, as it has enabled the wider investigation of the phase space topology, and features heavily in this thesis. Indeed,  the numerical schemes used within are formidable, and certainly beyond my ability to recreate within the thesis timescale, though I shall outline them in Section~\ref{sec:channelflow}. Given these tools, then, we can imagine that it may be possible to construct a web of periodic orbits, equilibria and their heteroclinc connections, and then predict to some accuracy the long-term dynamical behavior, based on transitions between these different states. 
	
	\chapter{The First}
    	This is the first page of the first chapter. You may delete the contents of this chapter so you can add your own text; it's just here to show you some examples. 
	
\section{References, Labels, Custom Commands and Footnotes}
It is easy to refer to anything within your document using the \texttt{label} and \texttt{ref} tags.  Labels must be unique and shouldn't use any odd characters; generally sticking to letters and numbers (no spaces) should be fine. Put the label on whatever you want to refer to, and put the reference where you want the reference. \LaTeX\ will keep track of the chapter, section, and figure or table numbers for you. 

\subsection{References and Labels}
Sometimes you'd like to refer to a table or figure, e.g. you can see in Figure \ref{subd2} that you can rotate figures . Start by labeling your figure or table with the label command (\verb=\label{labelvariable}=) below the caption (see the chapter on graphics and tables for examples). Then when you would like to refer to the table or figure, use the ref command (\verb=\ref{labelvariable}=). Make sure your label variables are unique; you can't have two elements named ``default." Also, since the reference command only puts the figure or table number, you will have to put  ``Table" or ``Figure" as appropriate, as seen in the following examples:

 As I showed in Table \ref{inheritance} many factors can be assumed to follow from inheritance. Also see the Figure \ref{subd} for an illustration.
 
\subsection{Custom Commands}\label{commands}
Are you sick of writing the same complex equation or phrase over and over? 

The custom commands should be placed in the preamble, or at least prior to the first usage of the command. The structure of the \verb=\newcommand= consists of the name of the new command in curly braces, the number of arguments to be made in square brackets and then, inside a new set of curly braces, the command(s) that make up the new command. The whole thing is sandwiched inside a larger set of curly braces. 

% Note: you cannot use numbers in your commands!
\newcommand{\hydro}{H$_2$SO$_4$}

In other words, if you want to make a shorthand for H$_2$SO$_4$, which doesn't include an argument, you would write: \verb=\newcommand{\hydro}{H$_2$SO$_4$}= and then when you needed  to use the command you would type \verb=\hydro=. (sans verb and the equals sign brackets, if you're looking at the .tex version). For example: \hydro

\subsection{Footnotes and Endnotes}
	You might want to footnote something.\footnote{footnote text} Be sure to leave no spaces between the word immediately preceding the footnote command and the command itself. The footnote will be in a smaller font and placed appropriately. Endnotes work in much the same way. More information can be found about both on the CUS site.
	
\section{Bibliographies}
	Of course you will need to cite things, and you will probably accumulate an armful of sources. This is why BibTeX was created. For more information about BibTeX and bibliographies, see our CUS site (\url{web.reed.edu/cis/help/latex/index.html})\footnote{\cite{reedweb:2007}}. There are three pages on this topic: {\it bibtex} (which talks about using BibTeX, at \url{/latex/bibtex.html}), {\it bibtexstyles} (about how to find and use the bibliography style that best suits your needs, at \url{/latex/bibtexstyles.html}) and {\it bibman} (which covers how to make and maintain a bibliography by hand, without BibTeX, at at \url{/latex/bibman.html}). The last page will not be useful unless you have only a few sources. There used to be APA stuff here, but we don't need it since I've fixed this with my apa-good natbib style file.
	
\subsection{Tips for Bibliographies}
\begin{enumerate}
\item Like with thesis formatting, the sooner you start compiling your bibliography for something as large as thesis, the better. Typing in source after source is mind-numbing enough; do you really want to do it for hours on end in late April? Think of it as procrastination.
\item The cite key (a citation's label) needs to be unique from the other entries.
\item When you have more than one author or editor, you need to separate each author's name by the word ``and'' e.g.\\ \verb+Author = {Noble, Sam and Youngberg, Jessica},+.
\item Bibliographies made using BibTeX (whether manually or using a manager) accept LaTeX markup, so you can italicize and add symbols as necessary.
\item To force capitalization in an article title or where all lowercase is generally used, bracket the capital letter in curly braces.
\item You can add a Reed Thesis citation\footnote{\cite{noble:2002}} option. The best way to do this is to use the phdthesis type of citation, and use the optional ``type'' field to enter ``Reed thesis'' or ``Undergraduate thesis''. Here's a test of Chicago, showing the second cite in a row\footnote{\cite{noble:2002}} being different. Also the second time not in a row\footnote{\cite{reedweb:2007}} should be different. Of course in other styles they'll all look the same.
\end{enumerate}
\section{Anything else?}
If you'd like to see examples of other things in this template, please contact CUS (email cus@reed.edu) with your suggestions. We love to see people using \LaTeX\ for their theses, and are happy to help.

	\input{math}

	\chapter{Tables and Graphics}

\section{Tables}
	The following section contains examples of tables, most of which have been commented out for brevity. (They will show up in the .tex document in red, but not at all in the .pdf). For more help in constructing a table (or anything else in this document), please see the LaTeX pages on the CUS site. 

\begin{table}[htdp] % begins the table floating environment. This enables LaTeX to fit the table where it works best and lets you add a caption.
\caption[Basic Table 1]{A Basic Table: Correlation of Factors between Parents and Child, Showing Inheritance} 
% The words in square brackets of the caption command end up in the Table of Tables. The words in curly braces are the caption directly over the table.
\begin{center} 
% makes the table centered
\begin{tabular}{c c c c} 
% the tabular environment is used to make the table itself. The {c c c c} specify that the table will have four columns and they will all be center-aligned. You can make the cell contents left aligned by replacing the Cs with Ls or right aligned by using Rs instead. Add more letters for more columns, and pipes (the vertical line above the backslash) for vertical lines. Another useful type of column is the p{width} column, which forces text to wrap within whatever width you specify e.g. p{1in}. Text will wrap badly in narrow columns though, so beware.
\toprule % a horizontal line, slightly thicker than \hline, depends on the booktabs package
  Factors &  Correlation between Parents \& Child & Inherited \\ % the first row of the table. Separate columns with ampersands and end the line with two backslashes. An environment begun in one cell will not carry over to adjacent rows.
  \midrule % another horizontal line
Education & -0.49 & Yes \\ % another row
Socio-Economic Status & 0.28 & Slight \\
Income & 0.08 & No\\
Family Size & 0.19 & Slight \\
Occupational Prestige &0.21 & Slight \\
\bottomrule % yet another horizontal line
\end{tabular}
\end{center}
\label{inheritance} % labels are useful when you have more than one table or figure in your document. See our online documentation for more on this.
\end{table}

	\clearpage 
%% \clearpage ends the page, and also dumps out all floats. 
%% Floats are things like tables and figures.

If you want to make a table that is longer than a page, you will want to use the longtable environment. Uncomment the table below to see an example, or see our online documentation.

	\begin{longtable}{||c|c|c|c||}
	 	\caption[Long Table]{An example of a long table, with headers that repeat on each subsequent page: Results from the summers of 1998 and 1999 work at Reed College done
by Grace Brannigan, Robert Holiday and Lien Ngo in 1998 and Kate Brown and
Christina Inman in 1999.}\\ \hline
	    	  \multicolumn{4}{||c||}{Chromium Hexacarbonyl} \\\hline
		   State & Laser wavelength & Buffer gas & Ratio of $\frac{\textrm{Intensity
at vapor pressure}}{\textrm{Intensity at 240 Torr}}$ \\ \hline
		  \endfirsthead
		\hline     State & Laser wavelength & Buffer gas & Ratio of
$\frac{\textrm{Intensity at vapor pressure}}{\textrm{Intensity at 240 Torr}}$\\
\hline
		    \endhead

	    $z^{7}P^{\circ}_{4}$ & 266 nm & Argon & 1.5 \\\hline
	    $z^{7}P^{\circ}_{2}$ & 355 nm & Argon & 0.57 \\\hline
	    $y^{7}P^{\circ}_{3}$ & 266 nm & Argon & 1 \\\hline
	    $y^{7}P^{\circ}_{3}$ & 355 nm & Argon & 0.14 \\\hline
	    $y^{7}P^{\circ}_{2}$ & 355 nm & Argon & 0.14 \\\hline
	    $z^{5}P^{\circ}_{3}$ & 266 nm & Argon & 1.2 \\\hline
	    $z^{5}P^{\circ}_{3}$ & 355 nm & Argon & 0.04 \\\hline
	    $z^{5}P^{\circ}_{3}$ & 355 nm & Helium & 0.02 \\\hline
	    $z^{5}P^{\circ}_{2}$ & 355 nm & Argon & 0.07 \\\hline
	    $z^{5}P^{\circ}_{1}$ & 355 nm & Argon & 0.05 \\\hline
	    $y^{5}P^{\circ}_{3}$ & 355 nm & Argon & 0.05, 0.4 \\\hline
	    $y^{5}P^{\circ}_{3}$ & 355 nm & Helium & 0.25 \\\hline
	    $z^{5}F^{\circ}_{4}$ & 266 nm & Argon & 1.4 \\\hline
	    $z^{5}F^{\circ}_{4}$ & 355 nm & Argon & 0.29 \\\hline
	    $z^{5}F^{\circ}_{4}$ & 355 nm & Helium & 1.02 \\\hline
	    $z^{5}D^{\circ}_{4}$ & 355 nm & Argon & 0.3 \\\hline
	    $z^{5}D^{\circ}_{4}$ & 355 nm & Helium & 0.65 \\\hline
	    $y^{5}H^{\circ}_{7}$ & 266 nm & Argon & 0.17 \\\hline
	    $y^{5}H^{\circ}_{7}$ & 355 nm & Argon & 0.13 \\\hline
	    $y^{5}H^{\circ}_{7}$ & 355 nm & Helium & 0.11 \\\hline
	    $a^{5}D_{3}$ & 266 nm & Argon & 0.71 \\\hline
	    $a^{5}D_{2}$ & 266 nm & Argon & 0.77 \\\hline
	    $a^{5}D_{2}$ & 355 nm & Argon & 0.63 \\\hline
	    $a^{3}D_{3}$ & 355 nm & Argon & 0.05 \\\hline
	    $a^{5}S_{2}$ & 266 nm & Argon & 2 \\\hline
	    $a^{5}S_{2}$ & 355 nm & Argon & 1.5 \\\hline
	    $a^{5}G_{6}$ & 355 nm & Argon & 0.91 \\\hline
	    $a^{3}G_{4}$ & 355 nm & Argon & 0.08 \\\hline
	    $e^{7}D_{5}$ & 355 nm & Helium & 3.5 \\\hline
	    $e^{7}D_{3}$ & 355 nm & Helium & 3 \\\hline
	    $f^{7}D_{5}$ & 355 nm & Helium & 0.25 \\\hline
	    $f^{7}D_{5}$ & 355 nm & Argon & 0.25 \\\hline
	    $f^{7}D_{4}$ & 355 nm & Argon & 0.2 \\\hline
	    $f^{7}D_{4}$ & 355 nm & Helium & 0.3 \\\hline
	    \multicolumn{4}{||c||}{Propyl-ACT} \\\hline
%	    State & Laser wavelength & Buffer gas & Ratio of $\frac{\textrm{Intensity
%at vapor pressure}}{\textrm{Intensity at 240 Torr}}$\\ \hline
	    $z^{7}P^{\circ}_{4}$ & 355 nm & Argon & 1.5 \\\hline
	    $z^{7}P^{\circ}_{3}$ & 355 nm & Argon & 1.5 \\\hline
	    $z^{7}P^{\circ}_{2}$ & 355 nm & Argon & 1.25 \\\hline
	    $z^{7}F^{\circ}_{5}$ & 355 nm & Argon & 2.85 \\\hline
	    $y^{7}P^{\circ}_{4}$ & 355 nm & Argon & 0.07 \\\hline
	    $y^{7}P^{\circ}_{3}$ & 355 nm & Argon & 0.06 \\\hline
	    $z^{5}P^{\circ}_{3}$ & 355 nm & Argon & 0.12 \\\hline
	    $z^{5}P^{\circ}_{2}$ & 355 nm & Argon & 0.13 \\\hline
	    $z^{5}P^{\circ}_{1}$ & 355 nm & Argon & 0.14 \\\hline
	    \multicolumn{4}{||c||}{Methyl-ACT} \\\hline
%	    State & Laser wavelength & Buffer gas & Ratio of $\frac{\textrm{Intensity
% at vapor pressure}}{\textrm{Intensity at 240 Torr}}$\\ \hline
	    $z^{7}P^{\circ}_{4}$ & 355 nm & Argon & 1.6, 2.5 \\\hline
	    $z^{7}P^{\circ}_{4}$ & 355 nm & Helium & 3 \\\hline
	    $z^{7}P^{\circ}_{4}$ & 266 nm & Argon & 1.33 \\\hline
	    $z^{7}P^{\circ}_{3}$ & 355 nm & Argon & 1.5 \\\hline
	    $z^{7}P^{\circ}_{2}$ & 355 nm & Argon & 1.25, 1.3 \\\hline
	    $z^{7}F^{\circ}_{5}$ & 355 nm & Argon & 3 \\\hline
	    $y^{7}P^{\circ}_{4}$ & 355 nm & Argon & 0.07, 0.08 \\\hline
	    $y^{7}P^{\circ}_{4}$ & 355 nm & Helium & 0.2 \\\hline
	    $y^{7}P^{\circ}_{3}$ & 266 nm & Argon & 1.22 \\\hline
	    $y^{7}P^{\circ}_{3}$ & 355 nm & Argon & 0.08 \\\hline
	    $y^{7}P^{\circ}_{2}$ & 355 nm & Argon & 0.1 \\\hline
	    $z^{5}P^{\circ}_{3}$ & 266 nm & Argon & 0.67 \\\hline
	    $z^{5}P^{\circ}_{3}$ & 355 nm & Argon & 0.08, 0.17 \\\hline
	    $z^{5}P^{\circ}_{3}$ & 355 nm & Helium & 0.12 \\\hline
	    $z^{5}P^{\circ}_{2}$ & 355 nm & Argon & 0.13 \\\hline
	    $z^{5}P^{\circ}_{1}$ & 355 nm & Argon & 0.09 \\\hline
	    $y^{5}H^{\circ}_{7}$ & 355 nm & Argon & 0.06, 0.05 \\\hline
	    $a^{5}D_{3}$ & 266 nm & Argon & 2.5 \\\hline
	    $a^{5}D_{2}$ & 266 nm & Argon & 1.9 \\\hline
	    $a^{5}D_{2}$ & 355 nm & Argon & 1.17 \\\hline
	    $a^{5}S_{2}$ & 266 nm & Argon & 2.3 \\\hline
	    $a^{5}S_{2}$ & 355 nm & Argon & 1.11 \\\hline
	    $a^{5}G_{6}$ & 355 nm & Argon & 1.6 \\\hline
	    $e^{7}D_{5}$ & 355 nm & Argon & 1 \\\hline

		\end{longtable}

   
   \section{Figures}
   
	If your thesis has a lot of figures, \LaTeX\ might behave better for you than that other word processor.  One thing that may be annoying is the way it handles ``floats'' like tables and figures. \LaTeX\ will try to find the best place to put your object based on the text around it and until you're really, truly done writing you should just leave it where it lies.   There are some optional arguments to the figure and table environments to specify where you want it to appear; see the comments in the first figure.

	If you need a graphic or tabular material to be part of the text, you can just put it inline. If you need it to appear in the list of figures or tables, it should be placed in the floating environment. 
	
	To get a figure from StatView, JMP, SPSS or other statistics program into a figure, you can print to pdf or save the image as a jpg or png. Precisely how you will do this depends on the program: you may need to copy-paste figures into Photoshop or other graphic program, then save in the appropriate format.
	
	Below we have put a few examples of figures. For more help using graphics and the float environment, see our online documentation.
	
	And this is how you add a figure with a graphic:
	\begin{figure}[h]
	% the options are h = here, t = top, b = bottom, p = page of figures.
	% you can add an exclamation mark to make it try harder, and multiple
	% options if you have an order of preference, e.g.
	% \begin{figure}[h!tbp]
	   
	       \centering
	    % DO NOT ADD A FILENAME EXTENSION TO THE GRAPHIC FILE
	    \includegraphics{subdivision}
	     \caption{A Figure}
	 \label{subd}
	\end{figure}

\clearpage %% starts a new page and stops trying to place floats such as tables and figures

\section{More Figure Stuff}
You can also scale and rotate figures.
 	\begin{figure}[h!]
	   
	       \centering
	    % DO NOT ADD A FILENAME EXTENSION TO THE GRAPHIC FILE
	    \includegraphics[scale=0.5,angle=180]{subdivision}
	    % if your figure shows up not where you want it, it may just be too big to fit. You can use the scale argument to shrink it, e.g. scale=0.85 is 85 percent of the original size. 
	     \caption{A Smaller Figure, Flipped Upside Down}
	 \label{subd2}
	\end{figure}

\section{Even More Figure Stuff}
With some clever work you can crop a figure, which is handy if (for instance) your EPS or PDF is a little graphic on a whole sheet of paper. The viewport arguments are the lower-left and upper-right coordinates for the area you want to crop.

 	\begin{figure}[h!]
	    	       \centering
	    % DO NOT ADD A FILENAME EXTENSION TO THE GRAPHIC FILE
	   \includegraphics[clip=true, viewport=.0in .0in 1in 1in]{subdivision}
	    \caption{A Cropped Figure}
	 \label{subd3}
	\end{figure}
	
      \subsection{Common Modifications}
      The following figure features the more popular changes thesis students want to their figures. This information is also on the web at \url{web.reed.edu/cis/help/latex/graphics.html}.
           \renewcommand{\thefigure}{0.\arabic{figure}} %Renumbers the figure to the type 0.x
    \addtocounter{figure}{4} %starts the figure numbering at 4
    \begin{figure}[htbp]
    \begin{center}
   \includegraphics[scale=0.5]{subdivision}
    \caption[Flower type and percent specialization]{\footnotesize{Interaction bar plot showing the degree of specialization for each flower type.}} %the special ToC caption is in square brackets. The \footnotesize makes the figure caption smaller
    \label{barplot}
    \end{center}
    \end{figure} 

	\input{conclusion}

%If you feel it necessary to include an appendix, it goes here.
	\appendix

	\chapter{The First Appendix}
An appendix full of awesome
	\chapter{The Second Appendix, for Fun}
An appendix full of win


%This is where endnotes are supposed to go, if you have them.
%I have no idea how endnotes work with LaTeX.

  \backmatter % backmatter makes the index and bibliography appear properly in the t.o.c...

% if you're using bibtex, the next line forces every entry in the bibtex file to be included
% in your bibliography, regardless of whether or not you've cited it in the thesis.
  \nocite{*}

% Rename my bibliography to be called "Works Cited" and not "References" or ``Bibliography''
% \renewcommand{\bibname}{Works Cited}

%    \bibliographystyle{bsts/mla-good} % there are a variety of styles available; 
%  \bibliographystyle{plainnat}
% replace ``plainnat'' with the style of choice. You can refer to files in the bsts or APA 
% subfolder, e.g. 
 \bibliographystyle{APA/apa-good}  % or
 \bibliography{thesis}
 % Comment the above two lines and uncomment the next line to use biblatex-chicago.
 %\printbibliography[heading=bibintoc]

% Finally, an index would go here... but it is also optional.
\end{document}
