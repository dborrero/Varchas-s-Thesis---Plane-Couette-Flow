\chapter*{Conclusion}
         \addcontentsline{toc}{chapter}{Conclusion}
	\chaptermark{Conclusion}
	\markboth{Conclusion}{Conclusion}
	\setcounter{chapter}{5}
	\setcounter{section}{0}
	\epigraph{
	Well-chosen, non-frivolous epigraphs can enhance a thesis.
	}{Michael Scott} 

\section{Summary}

The goal of this thesis has been to find and investigate numerically the properties of \ecs\ with broken symmetry. The hope was that these symmetry broken \ecs\ would better predict  turbulent dynamics, which itself has no symmetry at all. As a result, understanding symmetry broken \ecs\ may lead to a more fundamental understanding of turbulent dynamics. Four periodic orbits, nicknamed the Gang of Four, were found, with periods ranging from 8 to 85 time units, in fully and partially symmetric subspaces. The symmetric periodic orbits (P85 and P60) have no relative velocity and a low number of unstable eigenvectors. On the other hand, the asymmetric periodic orbits (P32 and P8) have some streamwise and spanwise relative velocity respectively, consistent with their broken streamwise and spanwise symmetry (\refTab{tab:summary}).  Projection of the Gang of Four onto the dissipation-energy input plane in \refFig{fig:turbDI} indicates that P8 likely has no influence on turbulent dynamics. However, as computational resources are limited and the cost of analysis scales linearly with the period, we were forced to choose P8 as the principle orbit of investigation. \\ 

Application of the Arnoldi iteration gave the unstable eigenvalues and corresponding eigenvectors of P8. Perturbing P8 along the most unstable eigenvector gives the trajectory in \refFig{fig:p8E1}. Computing its Poincare section gives the spiral structure seen in \refFig{fig:E1Poincare}, which is relatively consistent with theoretical behavior. Parametric continuation of P8 in the spanwise box length $L_Z$ results in behavior suggestive of a bifurcation, with two turning points at which the continuation algorithm switches from reducing $L_z$ and increasing $T$ to increasing $L_z$ and reducing $T$ and vice versa, shown in the S-shaped diagram in \refFig{fig:LZBif}. Analysis of the eigenvalues at both sides of the turning points (\refFigs{fig:P8FirstBifurcation}{fig:P8SecondBifurcation}) suggested that a manifold changed stability (from stable to unstable) in both cases, at a location that was consistent with the turning point.  In addition, fitting the DI projection of P8 at various box lengths both to the area and circumference of the orbit gave relations that show changes in its behavior at $L_z$ consistent with the turning points (\refFigs{fig:areaCell}{fig:circumCell}). Given the consistently strong indicators that significant structural changes occur, it seems reasonable to conclude that there is some local change in the state space that is consistent with the occurence of a bifurcation at the turning points $\mathfrak{Z}_1$ and $\mathfrak{Z}_2$.\\


\section{Future Work}

There are many threads of investigation we would have liked to have followed up on, time permitting. Most importantly, we would have investigated in more detail the behavior of the neglected members of the Gang of Four, which may have led in directions that we cannot anticipate. As far as P8 is concerned, preliminary data suggests that fully symmetric periodic orbits have a larger proportion of purely real unstable eigenvectors in comparison to partially symmetric periodic orbits. Since the behavior of a perturbation along a complex eigenvector would likely lead to more complex, chaotic behavior, it may be possible that trajectories near partially-symmetric orbits have, for instance, enhanced mixing properties when compared to trajectories near fully symmetric orbits. We would have also liked to have parametrically continued P8 in \ReN, since physical intuition suggests that it ought to appear from a bifurcation at some critical $\ReN < 400$.  It was also noted that when P8 was discovered from a guess generated by a random perturbation off of P60, the trajectory fell into its temporary orbit around P8 extremely quickly, despite the fact that P60 and P8 appear to live in distinct parts of the phase space of \refFigs{fig:POStateSpace}{fig:DIGOF}, and trajectories originating from near P60 in the DI projection appear to all move \emph{away} from P8. One possible explanation for this is that there exists some heteroclinic connection between P8 and P60 that guided the perturbation.  